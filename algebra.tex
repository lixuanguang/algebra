\documentclass[a4paper,sfsidenotes,openany]{tufte-book}

\hypersetup{colorlinks}% uncomment this line if you prefer colored hyperlinks (e.g., for onscreen viewing)

%%
% Book metadata
\title{Abstract Algebra\thanks{}}
\author[Arthur Li]{Arthur Li}
\publisher{}

%%
% If they're installed, use Bergamo and Chantilly from www.fontsite.com.
% They're clones of Bembo and Gill Sans, respectively.
%\IfFileExists{bergamo.sty}{\usepackage[osf]{bergamo}}{}% Bembo
%\IfFileExists{chantill.sty}{\usepackage{chantill}}{}% Gill Sans

%\usepackage{microtype}

%%
% Used to set numbering of the content
\setcounter{tocdepth}{1}
\setcounter{secnumdepth}{1}

%%
% Just some sample text
\usepackage{lipsum}

%%
% Math packages
\usepackage[utf8]{inputenc}
\usepackage{mathtools}
\usepackage{amssymb}
\usepackage{bm}
\usepackage{amsthm}
\usepackage{amsmath}
\usepackage{rotating}
\usepackage{mathrsfs}
\usepackage{tikz-cd}
\usepackage{float}

%%
% Math Commands
\def\upint{\mathchoice%
	{\mkern13mu\overline{\vphantom{\intop}\mkern7mu}\mkern-20mu}%
	{\mkern7mu\overline{\vphantom{\intop}\mkern7mu}\mkern-14mu}%
	{\mkern7mu\overline{\vphantom{\intop}\mkern7mu}\mkern-14mu}%
	{\mkern7mu\overline{\vphantom{\intop}\mkern7mu}\mkern-14mu}%
	\int}
\def\lowint{\mkern3mu\underline{\vphantom{\intop}\mkern7mu}\mkern-10mu\int}
\usepackage{tikz}
\newcommand{\N}{\mathbb{N}}
\newcommand{\R}{\mathbb{R}}
\newcommand{\Q}{\mathbb{Q}}
\newcommand{\x}{\mathbf{x}}
\newcommand{\F}{\mathbf{F}}
\newcommand{\f}{\mathbf{f}}
\newcommand{\y}{\mathbf{y}}
\renewcommand{\b}{\mathbf{b}}
\renewcommand{\c}{\mathbf{c}}
\renewcommand{\a}{\mathbf{a}}
\newcommand{\h}{\mathbf{h}}
\newcommand{\g}{\mathbf{g}}
\newcommand{\z}{\mathbf{z}}
\newcommand{\ze}{\mathbf{0}}
\newcommand{\Z}{\mathbb{Z}}
\newcommand{\norm}[1]{\left\lVert#1\right\rVert}
\newcommand{\abs}[1]{\left\lvert#1\right\rvert}
\newcommand{\brk}[1]{ \left[#1\right] }
\newcommand{\brc}[1]{ \left\{#1\right\} }
\newcommand{\paren}[1]{ \left(#1\right) }
\newcommand{\normop}[1]{\left\lVert#1\right\rVert_\text{op}}
\newcommand{\LL}{\mathcal{L}}
\newcommand{\uni}{\overset{\text{uni}}{\to}}
\DeclareMathOperator{\diam}{diam}
\newcommand{\Prr}[1]{\text{Pr}\left(#1\right)}


%%
% Set up math tools
\theoremstyle{definition}
\newtheorem{proposition}{Proposition}[section]
\newtheorem{theorem}{Theorem}[section]
\newtheorem{lemma}{Lemma}[section]
\newtheorem{definition}{Definition}[section]
\newtheorem{corollary}{Corollary}[section]
\newtheorem{ex}{Exercise}[section]
\newtheorem{note}{Notation}[section]
\newtheorem{example}{Example}[section]
\newtheorem{remark}{Remark}[section]

%%
% For nicely typeset tabular material
\usepackage{booktabs}


%%
% Formatting of chapter header
\titleformat{\chapter}
  [block]% shape
  {\relax\ifthenelse{\NOT\boolean{@tufte@symmetric}}{\begin{fullwidth}}{}}% format applied to label+text
  {\itshape\huge\thechapter}% label
  {1em}% horizontal separation between label and title body
  {\huge\rmfamily\itshape}% before the title body
  [\ifthenelse{\NOT\boolean{@tufte@symmetric}}{\end{fullwidth}}{}]% after the title body

%%
% For graphics / images
\usepackage{graphicx}
\setkeys{Gin}{width=\linewidth,totalheight=\textheight,keepaspectratio}
\graphicspath{{graphics/}}

% The fancyvrb package lets us customize the formatting of verbatim
% environments.  We use a slightly smaller font.
\usepackage{fancyvrb}
\fvset{fontsize=\normalsize}

%%
% Prints argument within hanging parentheses (i.e., parentheses that take
% up no horizontal space).  Useful in tabular environments.
\newcommand{\hangp}[1]{\makebox[0pt][r]{(}#1\makebox[0pt][l]{)}}

%%
% Prints an asterisk that takes up no horizontal space.
% Useful in tabular environments.
\newcommand{\hangstar}{\makebox[0pt][l]{*}}

%%
% Prints a trailing space in a smart way.
\usepackage{xspace}

%%
% Some shortcuts for Tufte's book titles.  The lowercase commands will
% produce the initials of the book title in italics.  The all-caps commands
% will print out the full title of the book in italics.
\newcommand{\vdqi}{\textit{VDQI}\xspace}
\newcommand{\ei}{\textit{EI}\xspace}
\newcommand{\ve}{\textit{VE}\xspace}
\newcommand{\be}{\textit{BE}\xspace}
\newcommand{\VDQI}{\textit{The Visual Display of Quantitative Information}\xspace}
\newcommand{\EI}{\textit{Envisioning Information}\xspace}
\newcommand{\VE}{\textit{Visual Explanations}\xspace}
\newcommand{\BE}{\textit{Beautiful Evidence}\xspace}

\newcommand{\TL}{Tufte-\LaTeX\xspace}

% Prints the month name (e.g., January) and the year (e.g., 2008)
\newcommand{\monthyear}{%
  \ifcase\month\or January\or February\or March\or April\or May\or June\or
  July\or August\or September\or October\or November\or
  December\fi\space\number\year
}


% Prints an epigraph and speaker in sans serif, all-caps type.
\newcommand{\openepigraph}[2]{%
  %\sffamily\fontsize{14}{16}\selectfont
  \begin{fullwidth}
  \sffamily\large
  \begin{doublespace}
  \noindent\allcaps{#1}\\% epigraph
  \noindent\allcaps{#2}% author
  \end{doublespace}
  \end{fullwidth}
}

% Inserts a blank page
\newcommand{\blankpage}{\newpage\hbox{}\thispagestyle{empty}\newpage}

\usepackage{units}

% Typesets the font size, leading, and measure in the form of 10/12x26 pc.
\newcommand{\measure}[3]{#1/#2$\times$\unit[#3]{pc}}

% Macros for typesetting the documentation
\newcommand{\hlred}[1]{\textcolor{Maroon}{#1}}% prints in red
\newcommand{\hangleft}[1]{\makebox[0pt][r]{#1}}
\newcommand{\hairsp}{\hspace{1pt}}% hair space
\newcommand{\hquad}{\hskip0.5em\relax}% half quad space
\newcommand{\TODO}{\textcolor{red}{\bf TODO!}\xspace}
\newcommand{\na}{\quad--}% used in tables for N/A cells
\providecommand{\XeLaTeX}{X\lower.5ex\hbox{\kern-0.15em\reflectbox{E}}\kern-0.1em\LaTeX}
\newcommand{\tXeLaTeX}{\XeLaTeX\index{XeLaTeX@\protect\XeLaTeX}}
% \index{\texttt{\textbackslash xyz}@\hangleft{\texttt{\textbackslash}}\texttt{xyz}}
\newcommand{\tuftebs}{\symbol{'134}}% a backslash in tt type in OT1/T1
\newcommand{\doccmdnoindex}[2][]{\texttt{\tuftebs#2}}% command name -- adds backslash automatically (and doesn't add cmd to the index)
\newcommand{\doccmddef}[2][]{%
  \hlred{\texttt{\tuftebs#2}}\label{cmd:#2}%
  \ifthenelse{\isempty{#1}}%
    {% add the command to the index
      \index{#2 command@\protect\hangleft{\texttt{\tuftebs}}\texttt{#2}}% command name
    }%
    {% add the command and package to the index
      \index{#2 command@\protect\hangleft{\texttt{\tuftebs}}\texttt{#2} (\texttt{#1} package)}% command name
      \index{#1 package@\texttt{#1} package}\index{packages!#1@\texttt{#1}}% package name
    }%
}% command name -- adds backslash automatically
\newcommand{\doccmd}[2][]{%
  \texttt{\tuftebs#2}%
  \ifthenelse{\isempty{#1}}%
    {% add the command to the index
      \index{#2 command@\protect\hangleft{\texttt{\tuftebs}}\texttt{#2}}% command name
    }%
    {% add the command and package to the index
      \index{#2 command@\protect\hangleft{\texttt{\tuftebs}}\texttt{#2} (\texttt{#1} package)}% command name
      \index{#1 package@\texttt{#1} package}\index{packages!#1@\texttt{#1}}% package name
    }%
}% command name -- adds backslash automatically
\newcommand{\docopt}[1]{\ensuremath{\langle}\textrm{\textit{#1}}\ensuremath{\rangle}}% optional command argument
\newcommand{\docarg}[1]{\textrm{\textit{#1}}}% (required) command argument
\newenvironment{docspec}{\begin{quotation}\ttfamily\parskip0pt\parindent0pt\ignorespaces}{\end{quotation}}% command specification environment
\newcommand{\docenv}[1]{\texttt{#1}\index{#1 environment@\texttt{#1} environment}\index{environments!#1@\texttt{#1}}}% environment name
\newcommand{\docenvdef}[1]{\hlred{\texttt{#1}}\label{env:#1}\index{#1 environment@\texttt{#1} environment}\index{environments!#1@\texttt{#1}}}% environment name
\newcommand{\docpkg}[1]{\texttt{#1}\index{#1 package@\texttt{#1} package}\index{packages!#1@\texttt{#1}}}% package name
\newcommand{\doccls}[1]{\texttt{#1}}% document class name
\newcommand{\docclsopt}[1]{\texttt{#1}\index{#1 class option@\texttt{#1} class option}\index{class options!#1@\texttt{#1}}}% document class option name
\newcommand{\docclsoptdef}[1]{\hlred{\texttt{#1}}\label{clsopt:#1}\index{#1 class option@\texttt{#1} class option}\index{class options!#1@\texttt{#1}}}% document class option name defined
\newcommand{\docmsg}[2]{\bigskip\begin{fullwidth}\noindent\ttfamily#1\end{fullwidth}\medskip\par\noindent#2}
\newcommand{\docfilehook}[2]{\texttt{#1}\index{file hooks!#2}\index{#1@\texttt{#1}}}
\newcommand{\doccounter}[1]{\texttt{#1}\index{#1 counter@\texttt{#1} counter}}

% Generates the index
\usepackage{makeidx}
\makeindex

\begin{document}

% Front matter
\frontmatter

\maketitle

\bibliographystyle{chicago}

\chapter*{Introduction}
\newthought
{This collection} of notes serve as a guide to mastering abstract algebra with content from undergraduate to graduate level course. The notes combine knowledge from different sources, including course notes and textbooks used in the courses.
	
\subsection{Prerequisites}
These notes will assume no familiarity with any aspects of abstract algebra, and builds upon the foundation from Group Theory to more abstract topics such as Categories and Commutative Algebra. A good starting point will be the series on  \href{https://www.youtube.com/watch?v=UwTQdOop-nU&list=PLwV-9DG53NDxU337smpTwm6sef4x-SCLv}{Visual Group Theory by Professor Matthew Macauley}.
	
Familiarity with basic styles of proof is assumed (contradiction, contrapositive, etc.).
	
\subsection{Organization and Sources}
This section will be edited as the notes progress towards completion.

\tableofcontents

%%%
% Start the main matter (normal chapters)
\mainmatter

% Chapter 1: Preliminaries
\chapter{Preliminaries}
\section{Introductory Ideas and Definitions}
\begin{fullwidth}

\begin{definition}
\textit{{\color{blue} Class}} is a collection $ A $ of objects (elements) such that given any object $ x $ it is possible to determine if $ x $ is a member of $ A $.\\
\end{definition}
\>

\begin{definition}
\textit{{\color{blue} Axiom of extensionality}} asserts that two classes with the same elements are equal. \\
(Formally, $ \left[x \in A \Longleftrightarrow x \in B \right] \Rightarrow A=B $).\\
\end{definition}
\>

\begin{definition}
A class is defined to be a \textit{{\color{blue} set}} if and only if there exists a class $ B $ such that $ A \in B $.\\
A class that is not a set is called a \textit{{\color{blue} proper set}}.\\
\end{definition}
\>

\begin{definition}
\textit{{\color{blue} Axiom of class formation}} asserts that for any statement $ P(y) $ in the first predicate calculus involve a variable $ y $, there exists a class $ A $ such that $ x \in A $ if and only if $ x $ is a set and the statement $ P(x) $ is true. The class is denoted 	$\{ x | P(x) \}$.\\
\end{definition}
\>

\begin{definition}
A class $ A $ is a \textit{{\color{blue} subclass}} of class $ B $ ($ B \subset A $) provided $ \forall x \in A, x \in A \Longleftrightarrow x \in B $. \\
A subclass $ A $ of a class $ B $ that is itself a set is called a \textit{{\color{blue} subset}} of $ B $.\\
The \textit{{\color{blue} empty or null set}} (denoted $\emptyset$) is the set with no elements.\\
\end{definition}
\>

\begin{definition}
\textit{{\color{blue} Power axiom}} asserts that for every set $ A $ the class $ P(A) $ of all subsets of $ A $ is itself a set. $ P(A) $ is the \textit{{\color{blue} power set}} of $ A $, denoted $ 2^A $.\\
\end{definition}
\>

\begin{definition}
A \textit{{\color{blue} family of sets}} indexed by (nonempty) class $ I $ is a collection of sets $ A_i $, one for each $ i \in I $ (denoted $\{ A_i | i \in I \}$).\newline
The \textit{{\color{blue} union}} is defined as $ \bigcup\limits_{i \in I}A_{i} = \{ x | x \in A_i \ for \ some \ i \in I \}$.\newline
The \textit{{\color{blue} intersection}} is defined as $ \bigcap\limits_{i \in I}A_{i} = \{ x | x \in A_i \ for \ every \ i \in I \}$.\\
If $ A \cap B = \emptyset $, then $ A $ and $ B $ are disjoint.\\
\end{definition}
\>

\begin{definition}
The \textit{{\color{blue} relative complement}} of $ A $ in $ B $ is the following subclass of $ B $: $ B-A = \{ x | x \in B \ and \ x \notin A \}$. \\
If all classes under discussion are subsets of some fixed set $ U $ (the universe of discussion), then $ U - A = A' $ is the \textit{{\color{blue} complement}} of $ A $.\\
\end{definition}
\>

\begin{definition}
Given classes $ A $ and $ B $, a \textit{{\color{blue} function / map / mapping}} $ f $ from $ A $ to $ B $ (written $ f: A \rightarrow B $ assigns to each $ a \in A $ exactly one element $ b \in B $.\newline
Then $ b $ is the value of function at $ a $, or the \textit{{\color{blue} image}} of $ a $, written $ f(a) $.\newline
$ A $ is the \textit{{\color{blue} domain}} of the function, written $ dom f $, and $ B $ is the \textit{{\color{blue} range}} or \textit{{\color{blue} codomain}}.\\
Two functions are \textit{{\color{blue} equal}} if they have the same domain and range, and have the same value for each element of their common domain.\\
\end{definition}
\>

\begin{definition}
If $ f: A \rightarrow B $ is a function and $ S \subset A$, the function from $ S $ to $ B $ given by $ a \mapsto f(a) $, for $ a \in S $, is \textit{{\color{blue} restriction}} of $ f $ to $ S $, denoted $ f|S: S \rightarrow B$.\newline
If $ S \in A$, the function $ 1_A | S: S \rightarrow A $ is the \textit{{\color{blue} inclusion map}} of $S$ into $A$.\\
\end{definition}
\>

\begin{definition}
Let $ f: A \rightarrow B $ and $ g: B \rightarrow C $ be functions. The \textit{{\color{blue} composite}} of $ f $ and $ g $ is the function $ A \rightarrow C $ given by $ a \mapsto g(f(a)), a \in A $. This is denoted $ g \circ f $ or simply $ gf $.\\
\end{definition}
\>

\begin{definition}
The \textit{{\color{blue} diagram of functions}} is said to be commutative if $ gf = h $, or if $ kh = gf $.\\

\begin{equation}\label{diagram}
\begin{tikzcd}
A \arrow{rr}{f} \arrow[swap]{dr}{h} & & B \arrow{dl}{g} \\[10pt]
    & C
\end{tikzcd}
\quad
\begin{tikzcd}[row sep=2.5em]
 A \ar{r}{f} \ar{d}{h} & B \ar{d}{g} \\
 C \ar{r}{k} & D
\end{tikzcd}
\end{equation}
\end{definition}
\>

\begin{definition}
Let $ f: A \rightarrow B $ be a function. If $ S \in A $, \textit{{\color{blue} the image of $ S $ under $ f $}} (denoted $ f(S)) $) is the class $ \{ b \in B | b=f(a) \ for \ some \ a \in S\} $.\\
The class $ f(A) $ is the \textit{{\color{blue} image of $ f $}}, denoted $ Im \ f $.\\
If $ T \subset B $, the \textit{{\color{blue} inverse image of $ T $}} under $ f $ (denoted $ f \textsuperscript{-1} (T) $), is the class $ \{ a \in A | f(a) \in T\} $.\\
\end{definition}
\>

\begin{definition}
A function $ f: A \rightarrow B$ is said to be \textit{{\color{blue} injective}} (or one-to-one) provided $ \forall a, \ a' \in A, \ a \neq a' \Rightarrow f(a) \neq f(a') $, or $ f(a) = f(a') \Rightarrow a = a'  $.\\
A function $ f $ is \textit{{\color{blue} surjective}} (or on-to) provided $ f(A) \approx B $; in other words, for each $ b \in B $, $ b=f(a) $ for some $ a \in A $.\\
A function $ f $ is \textit{{\color{blue} bijective}} (or one-to-one correspondence) if it is both injective and surjective.\\
\end{definition}
\>

\begin{definition}
The map $ g: B \rightarrow A $ is a \textit{{\color{blue} left inverse}} of $ f $ if $ gf = 1_A $.\\
The map $ h: B \rightarrow A $ is a \textit{{\color{blue} right inverse}} of $ f $ if $ fb = 1_B $.\\
If a map $ f: A \rightarrow B $ has both a left inverse $ g $ and a right inverse $ h $, then $ g = g1_B = g(fh) = (gf)h = 1 \textsubscript{A} h = h $, and $ g=h $ is the \textit{{\color{blue} two-sided inverse}}.\\
\end{definition}
\>

\end{fullwidth}


\newpage

% Chapter 2: Group Theory
\chapter{Group Theory}
\begin{fullwidth}
\section{Basic Axioms}

\section{Homomorphisms and Subgroups}

\section{Cyclic Groups}

\section{Cosets}

\section{Normality, Quotient Groups}

\section{Isomorphism Theorems}

\section{Symmetric, Alternating and Dihedral Groups}

\section{Categories, Products, Coproducts, Free Objects}

\section{Direct Products, Direct Sums}

\section{Free Groups, Free Products}

\section{Matrix Groups}
	
\end{fullwidth}

\newpage

% Chapter 3: Group Structures
\chapter{Group Structures}
\begin{fullwidth}
\section{Free Abelian Groups}

\section{Finitely Generated Abelian Groups}

\section{Krull-Schmidt Theorem}

\section{Group Action}

\section{The Sylow Theorems}

\section{Semidirect Products}

\section{Normal and Subnormal Series}

\end{fullwidth}

\newpage

% Chapter 4: Ring Theory
\chapter{Ring Theory}
\begin{fullwidth}
\section{Basic Axioms}

\section{Ring Homomorphisms}

\section{Ring Isomorphisms}

\section{Ideals, Rings of Fractions, Local Rings}

\section{Euclidean Domains, PID, UFD}

\end{fullwidth}

\newpage

% Chapter 5: Modules
\chapter{Modules}
\begin{fullwidth}
\section{Basic Axioms}

\end{fullwidth}

\newpage

% Chapter 6: Category Theory
\chapter{Category Theory}
\begin{fullwidth}
\section{Basic Axioms}

\end{fullwidth}

\newpage

\printindex

\end{document}