\documentclass[a4paper,sfsidenotes,openany]{tufte-book}

\hypersetup{colorlinks}% uncomment this line if you prefer colored hyperlinks (e.g., for onscreen viewing)

%%
% Book metadata
\title{Abstract Algebra\thanks{}}
\author[Arthur Li]{Arthur Li}
\publisher{}

%%
% If they're installed, use Bergamo and Chantilly from www.fontsite.com.
% They're clones of Bembo and Gill Sans, respectively.
%\IfFileExists{bergamo.sty}{\usepackage[osf]{bergamo}}{}% Bembo
%\IfFileExists{chantill.sty}{\usepackage{chantill}}{}% Gill Sans

%\usepackage{microtype}

%%
% Used to set numbering of the content
\setcounter{tocdepth}{1}
\setcounter{secnumdepth}{1}

%%
% Just some sample text
\usepackage{lipsum}

%%
% Math packages
\usepackage[utf8]{inputenc}
\usepackage{mathtools}
\usepackage{amssymb}
\usepackage{bm}
\usepackage{amsthm}
\usepackage{amsmath}
\usepackage{rotating}
\usepackage{mathrsfs}
\usepackage{tikz-cd}
\usepackage{float}
\usepackage{enumitem}

%%
% Math Commands
\def\upint{\mathchoice%
	{\mkern13mu\overline{\vphantom{\intop}\mkern7mu}\mkern-20mu}%
	{\mkern7mu\overline{\vphantom{\intop}\mkern7mu}\mkern-14mu}%
	{\mkern7mu\overline{\vphantom{\intop}\mkern7mu}\mkern-14mu}%
	{\mkern7mu\overline{\vphantom{\intop}\mkern7mu}\mkern-14mu}%
	\int}
\def\lowint{\mkern3mu\underline{\vphantom{\intop}\mkern7mu}\mkern-10mu\int}
\usepackage{tikz}
\newcommand{\N}{\mathbb{N}}
\newcommand{\R}{\mathbb{R}}
\newcommand{\Q}{\mathbb{Q}}
\newcommand{\x}{\mathbf{x}}
\newcommand{\F}{\mathbf{F}}
\newcommand{\f}{\mathbf{f}}
\newcommand{\y}{\mathbf{y}}
\renewcommand{\b}{\mathbf{b}}
\renewcommand{\c}{\mathbf{c}}
\renewcommand{\a}{\mathbf{a}}
\newcommand{\h}{\mathbf{h}}
\newcommand{\g}{\mathbf{g}}
\newcommand{\z}{\mathbf{z}}
\newcommand{\ze}{\mathbf{0}}
\newcommand{\Z}{\mathbb{Z}}
\newcommand{\norm}[1]{\left\lVert#1\right\rVert}
\newcommand{\abs}[1]{\left\lvert#1\right\rvert}
\newcommand{\brk}[1]{ \left[#1\right] }
\newcommand{\brc}[1]{ \left\{#1\right\} }
\newcommand{\paren}[1]{ \left(#1\right) }
\newcommand{\normop}[1]{\left\lVert#1\right\rVert_\text{op}}
\newcommand{\LL}{\mathcal{L}}
\newcommand{\uni}{\overset{\text{uni}}{\to}}
\DeclareMathOperator{\diam}{diam}
\newcommand{\Prr}[1]{\text{Pr}\left(#1\right)}

%%
% Set up math tools
\theoremstyle{theorem} 
\newtheorem{theorem}{Theorem}[section]
\newtheorem{corollary}[theorem]{Corollary}
\newtheorem{lemma}[theorem]{Lemma}
\newtheorem{proposition}[theorem]{Proposition}
\newtheorem{definition}[theorem]{Definition}
\newtheorem{example}[theorem]{Example}
\newtheorem{remark}[theorem]{Remark}

%%
% For nicely typeset tabular material
\usepackage{booktabs}


%%
% Formatting of chapter header
\titleformat{\chapter}
  [block]% shape
  {\relax\ifthenelse{\NOT\boolean{@tufte@symmetric}}{\begin{fullwidth}}{}}% format applied to label+text
  {\itshape\huge\thechapter}% label
  {1em}% horizontal separation between label and title body
  {\huge\rmfamily\itshape}% before the title body
  [\ifthenelse{\NOT\boolean{@tufte@symmetric}}{\end{fullwidth}}{}]% after the title body

%%
% For graphics / images
\usepackage{graphicx}
\setkeys{Gin}{width=\linewidth,totalheight=\textheight,keepaspectratio}
\graphicspath{{graphics/}}

% The fancyvrb package lets us customize the formatting of verbatim
% environments.  We use a slightly smaller font.
\usepackage{fancyvrb}
\fvset{fontsize=\normalsize}

%%
% Prints argument within hanging parentheses (i.e., parentheses that take
% up no horizontal space).  Useful in tabular environments.
\newcommand{\hangp}[1]{\makebox[0pt][r]{(}#1\makebox[0pt][l]{)}}

%%
% Prints an asterisk that takes up no horizontal space.
% Useful in tabular environments.
\newcommand{\hangstar}{\makebox[0pt][l]{*}}

%%
% Prints a trailing space in a smart way.
\usepackage{xspace}

%%
% Some shortcuts for Tufte's book titles.  The lowercase commands will
% produce the initials of the book title in italics.  The all-caps commands
% will print out the full title of the book in italics.
\newcommand{\vdqi}{\textit{VDQI}\xspace}
\newcommand{\ei}{\textit{EI}\xspace}
\newcommand{\ve}{\textit{VE}\xspace}
\newcommand{\be}{\textit{BE}\xspace}
\newcommand{\VDQI}{\textit{The Visual Display of Quantitative Information}\xspace}
\newcommand{\EI}{\textit{Envisioning Information}\xspace}
\newcommand{\VE}{\textit{Visual Explanations}\xspace}
\newcommand{\BE}{\textit{Beautiful Evidence}\xspace}

\newcommand{\TL}{Tufte-\LaTeX\xspace}

% Prints the month name (e.g., January) and the year (e.g., 2008)
\newcommand{\monthyear}{%
  \ifcase\month\or January\or February\or March\or April\or May\or June\or
  July\or August\or September\or October\or November\or
  December\fi\space\number\year
}


% Prints an epigraph and speaker in sans serif, all-caps type.
\newcommand{\openepigraph}[2]{%
  %\sffamily\fontsize{14}{16}\selectfont
  \begin{fullwidth}
  \sffamily\large
  \begin{doublespace}
  \noindent\allcaps{#1}\\% epigraph
  \noindent\allcaps{#2}% author
  \end{doublespace}
  \end{fullwidth}
}

% Inserts a blank page
\newcommand{\blankpage}{\newpage\hbox{}\thispagestyle{empty}\newpage}

\usepackage{units}

% Typesets the font size, leading, and measure in the form of 10/12x26 pc.
\newcommand{\measure}[3]{#1/#2$\times$\unit[#3]{pc}}

% Macros for typesetting the documentation
\newcommand{\hlred}[1]{\textcolor{Maroon}{#1}}% prints in red
\newcommand{\hangleft}[1]{\makebox[0pt][r]{#1}}
\newcommand{\hairsp}{\hspace{1pt}}% hair space
\newcommand{\hquad}{\hskip0.5em\relax}% half quad space
\newcommand{\TODO}{\textcolor{red}{\bf TODO!}\xspace}
\newcommand{\na}{\quad--}% used in tables for N/A cells
\providecommand{\XeLaTeX}{X\lower.5ex\hbox{\kern-0.15em\reflectbox{E}}\kern-0.1em\LaTeX}
\newcommand{\tXeLaTeX}{\XeLaTeX\index{XeLaTeX@\protect\XeLaTeX}}
% \index{\texttt{\textbackslash xyz}@\hangleft{\texttt{\textbackslash}}\texttt{xyz}}
\newcommand{\tuftebs}{\symbol{'134}}% a backslash in tt type in OT1/T1
\newcommand{\doccmdnoindex}[2][]{\texttt{\tuftebs#2}}% command name -- adds backslash automatically (and doesn't add cmd to the index)
\newcommand{\doccmddef}[2][]{%
  \hlred{\texttt{\tuftebs#2}}\label{cmd:#2}%
  \ifthenelse{\isempty{#1}}%
    {% add the command to the index
      \index{#2 command@\protect\hangleft{\texttt{\tuftebs}}\texttt{#2}}% command name
    }%
    {% add the command and package to the index
      \index{#2 command@\protect\hangleft{\texttt{\tuftebs}}\texttt{#2} (\texttt{#1} package)}% command name
      \index{#1 package@\texttt{#1} package}\index{packages!#1@\texttt{#1}}% package name
    }%
}% command name -- adds backslash automatically
\newcommand{\doccmd}[2][]{%
  \texttt{\tuftebs#2}%
  \ifthenelse{\isempty{#1}}%
    {% add the command to the index
      \index{#2 command@\protect\hangleft{\texttt{\tuftebs}}\texttt{#2}}% command name
    }%
    {% add the command and package to the index
      \index{#2 command@\protect\hangleft{\texttt{\tuftebs}}\texttt{#2} (\texttt{#1} package)}% command name
      \index{#1 package@\texttt{#1} package}\index{packages!#1@\texttt{#1}}% package name
    }%
}% command name -- adds backslash automatically
\newcommand{\docopt}[1]{\ensuremath{\langle}\textrm{\textit{#1}}\ensuremath{\rangle}}% optional command argument
\newcommand{\docarg}[1]{\textrm{\textit{#1}}}% (required) command argument
\newenvironment{docspec}{\begin{quotation}\ttfamily\parskip0pt\parindent0pt\ignorespaces}{\end{quotation}}% command specification environment
\newcommand{\docenv}[1]{\texttt{#1}\index{#1 environment@\texttt{#1} environment}\index{environments!#1@\texttt{#1}}}% environment name
\newcommand{\docenvdef}[1]{\hlred{\texttt{#1}}\label{env:#1}\index{#1 environment@\texttt{#1} environment}\index{environments!#1@\texttt{#1}}}% environment name
\newcommand{\docpkg}[1]{\texttt{#1}\index{#1 package@\texttt{#1} package}\index{packages!#1@\texttt{#1}}}% package name
\newcommand{\doccls}[1]{\texttt{#1}}% document class name
\newcommand{\docclsopt}[1]{\texttt{#1}\index{#1 class option@\texttt{#1} class option}\index{class options!#1@\texttt{#1}}}% document class option name
\newcommand{\docclsoptdef}[1]{\hlred{\texttt{#1}}\label{clsopt:#1}\index{#1 class option@\texttt{#1} class option}\index{class options!#1@\texttt{#1}}}% document class option name defined
\newcommand{\docmsg}[2]{\bigskip\begin{fullwidth}\noindent\ttfamily#1\end{fullwidth}\medskip\par\noindent#2}
\newcommand{\docfilehook}[2]{\texttt{#1}\index{file hooks!#2}\index{#1@\texttt{#1}}}
\newcommand{\doccounter}[1]{\texttt{#1}\index{#1 counter@\texttt{#1} counter}}

% Generates the index
\usepackage{makeidx}
\makeindex

\begin{document}

% Front matter
\frontmatter

\maketitle

\bibliographystyle{chicago}

\chapter*{Introduction}
\newthought
{This collection} of notes serve as a guide to mastering abstract algebra with content from undergraduate to graduate level course. The notes combine knowledge from different sources, including course notes and textbooks used in the courses.\\

The proofs for Theorems, Propositions and Lemmas will be added when I have completed the entire skeleton.
	
\subsection{Prerequisites}
These notes will assume no familiarity with any aspects of abstract algebra, and builds upon the foundation from Group Theory to more abstract topics such as Categories and Commutative Algebra. A good starting point will be the series on  \href{https://www.youtube.com/watch?v=UwTQdOop-nU&list=PLwV-9DG53NDxU337smpTwm6sef4x-SCLv}{Visual Group Theory by Professor Matthew Macauley}.
	
Familiarity with basic styles of proof is assumed (contradiction, contrapositive, etc.).
	
\subsection{Organization and Sources}
This section will be edited as the notes progress towards completion.

\tableofcontents

%%%
% Start the main matter (normal chapters)
\mainmatter

% Chapter 1: Preliminaries
\chapter{Preliminaries}
\section{Introductory Ideas and Definitions}
\begin{fullwidth}

\begin{definition}
\textit{{\color{blue} Class}} is a collection $ A $ of objects (elements) such that given any object $ x $ it is possible to determine if $ x $ is a member of $ A $.\\
\end{definition}
\>

\begin{definition}
\textit{{\color{blue} Axiom of extensionality}} asserts that two classes with the same elements are equal. \\
(Formally, $ \left[x \in A \Longleftrightarrow x \in B \right] \Rightarrow A=B $).\\
\end{definition}
\>

\begin{definition}
A class is defined to be a \textit{{\color{blue} set}} if and only if there exists a class $ B $ such that $ A \in B $.\\
A class that is not a set is called a \textit{{\color{blue} proper set}}.\\
\end{definition}
\>

\begin{definition}
\textit{{\color{blue} Axiom of class formation}} asserts that for any statement $ P(y) $ in the first predicate calculus involve a variable $ y $, there exists a class $ A $ such that $ x \in A $ if and only if $ x $ is a set and the statement $ P(x) $ is true. The class is denoted 	$\{ x | P(x) \}$.\\
\end{definition}
\>

\begin{definition}
A class $ A $ is a \textit{{\color{blue} subclass}} of class $ B $ ($ B \subset A $) provided $ \forall x \in A, x \in A \Longleftrightarrow x \in B $. \\
A subclass $ A $ of a class $ B $ that is itself a set is called a \textit{{\color{blue} subset}} of $ B $.\\
The \textit{{\color{blue} empty or null set}} (denoted $\emptyset$) is the set with no elements.\\
\end{definition}
\>

\begin{definition}
\textit{{\color{blue} Power axiom}} asserts that for every set $ A $ the class $ P(A) $ of all subsets of $ A $ is itself a set. $ P(A) $ is the \textit{{\color{blue} power set}} of $ A $, denoted $ 2^A $.\\
\end{definition}
\>

\begin{definition}
A \textit{{\color{blue} family of sets}} indexed by (nonempty) class $ I $ is a collection of sets $ A_i $, one for each $ i \in I $ (denoted $\{ A_i | i \in I \}$).\newline
The \textit{{\color{blue} union}} is defined as $ \bigcup\limits_{i \in I}A_{i} = \{ x | x \in A_i \ for \ some \ i \in I \}$.\newline
The \textit{{\color{blue} intersection}} is defined as $ \bigcap\limits_{i \in I}A_{i} = \{ x | x \in A_i \ for \ every \ i \in I \}$.\\
If $ A \cap B = \emptyset $, then $ A $ and $ B $ are disjoint.\\
\end{definition}
\>

\begin{definition}
The \textit{{\color{blue} relative complement}} of $ A $ in $ B $ is the following subclass of $ B $: $ B-A = \{ x | x \in B \ and \ x \notin A \}$.
If all classes under discussion are subsets of some fixed set $ U $ (the universe of discussion), then $ U - A = A' $ is the \textit{{\color{blue} complement}} of $ A $.\\
\end{definition}
\>

\begin{definition}
Given classes $ A $ and $ B $, a \textit{{\color{blue} function / map / mapping}} $ f $ from $ A $ to $ B $ (written $ f: A \rightarrow B $ assigns to each $ a \in A $ exactly one element $ b \in B $.\newline
Then $ b $ is the value of function at $ a $, or the \textit{{\color{blue} image}} of $ a $, written $ f(a) $.\newline
$ A $ is the \textit{{\color{blue} domain}} of the function, written $ dom f $, and $ B $ is the \textit{{\color{blue} range}} or \textit{{\color{blue} codomain}}.\\
Two functions are \textit{{\color{blue} equal}} if they have the same domain and range, and have the same value for each element of their common domain.\\
\end{definition}
\>

\begin{definition}
If $ f: A \rightarrow B $ is a function and $ S \subset A$, the function from $ S $ to $ B $ given by $ a \mapsto f(a) $, for $ a \in S $, is \textit{{\color{blue} restriction}} of $ f $ to $ S $, denoted $ f|S: S \rightarrow B$.\newline
If $ S \in A$, the function $ 1_A | S: S \rightarrow A $ is the \textit{{\color{blue} inclusion map}} of $S$ into $A$.\\
\end{definition}
\>

\begin{definition}
Let $ f: A \rightarrow B $ and $ g: B \rightarrow C $ be functions. The \textit{{\color{blue} composite}} of $ f $ and $ g $ is the function $ A \rightarrow C $ given by $ a \mapsto g(f(a)), a \in A $. This is denoted $ g \circ f $ or simply $ gf $.\\
\end{definition}
\>

\begin{definition}
The \textit{{\color{blue} diagram of functions}} is said to be commutative if $ gf = h $, or if $ kh = gf $.\\

\begin{equation}\label{diagram}
\begin{tikzcd}
A \arrow{rr}{f} \arrow[swap]{dr}{h} & & B \arrow{dl}{g} \\[10pt]
    & C
\end{tikzcd}
\quad
\begin{tikzcd}[row sep=2.5em]
 A \ar{r}{f} \ar{d}{h} & B \ar{d}{g} \\
 C \ar{r}{k} & D
\end{tikzcd}
\end{equation}
\end{definition}
\>

\begin{definition}
Let $ f: A \rightarrow B $ be a function. If $ S \in A $, \textit{{\color{blue} the image of $ S $ under $ f $}} (denoted $ f(S)) $) is the class $ \{ b \in B | b=f(a) \ for \ some \ a \in S\} $.\\
The class $ f(A) $ is the \textit{{\color{blue} image of $ f $}}, denoted $ Im \ f $.\\
If $ T \subset B $, the \textit{{\color{blue} inverse image of $ T $}} under $ f $ (denoted $ f \textsuperscript{-1} (T) $), is the class $ \{ a \in A | f(a) \in T\} $.\\
\end{definition}
\>

\begin{definition}
A function $ f: A \rightarrow B$ is said to be \textit{{\color{blue} injective}} (or one-to-one) provided $ \forall a, \ a' \in A, \ a \neq a' \Rightarrow f(a) \neq f(a') $, or $ f(a) = f(a') \Rightarrow a = a'  $.\\
A function $ f $ is \textit{{\color{blue} surjective}} (or on-to) provided $ f(A) \approx B $; in other words, for each $ b \in B $, $ b=f(a) $ for some $ a \in A $.\\
A function $ f $ is \textit{{\color{blue} bijective}} (or one-to-one correspondence) if it is both injective and surjective.\\
\end{definition}
\>

\begin{definition}
The map $ g: B \rightarrow A $ is a \textit{{\color{blue} left inverse}} of $ f $ if $ gf = 1_A $.\\
The map $ h: B \rightarrow A $ is a \textit{{\color{blue} right inverse}} of $ f $ if $ fb = 1_B $.\\
If a map $ f: A \rightarrow B $ has both a left inverse $ g $ and a right inverse $ h $, then $ g = g1_B = g(fh) = (gf)h = 1 \textsubscript{A} h = h $, and $ g=h $ is the \textit{{\color{blue} two-sided inverse}}.\\
\end{definition}
\>

\end{fullwidth}


\newpage

% Chapter 2: Group Theory
\chapter{Group Theory}
\begin{fullwidth}
\section{Basic Axioms}

\section{Homomorphisms and Subgroups}

\section{Cyclic Groups}

\section{Cosets}

\section{Normality, Quotient Groups}

\section{Isomorphism Theorems}

\section{Symmetric, Alternating and Dihedral Groups}

\section{Categories, Products, Coproducts, Free Objects}

\section{Direct Products, Direct Sums}

\section{Free Groups, Free Products}

\section{Matrix Groups}
	
\end{fullwidth}

\newpage

% Chapter 3: Group Structures
\chapter{Group Structures}
\begin{fullwidth}
\section{Free Abelian Groups}

\section{Finitely Generated Abelian Groups}

\section{Krull-Schmidt Theorem}

\section{Group Action}

\section{The Sylow Theorems}

\section{Semidirect Products}

\section{Normal and Subnormal Series}

\end{fullwidth}

\newpage

% Chapter 4: Ring Theory
\chapter{Ring Theory}
\begin{fullwidth}
\section{Basic Axioms}

\begin{definition}
A \textit{{\color{blue} ring}} is a nonempty set $ R $ with two binary operations $ + $ (addition) and $ \times $ (multiplication), $ (R, +, \times) $, such that:
\begin{enumerate}[label=(\roman*),leftmargin=0pt, itemindent=4em, align=left]
\item $(R, +)$ is an additive abelian group with $0$ as the additive identity
\item the binary operation $\times$ is associative: $(a \times b) \times c = a \times (b \times c) $, $ \forall a, b, c \in R$
\item left and right distributive laws: $(a+b) \times c = (a \times c) + (b \times c) $, $a \times (b+c) = (a \times c) + (b \times c)$, $\forall a, b, c \in R$.
\end{enumerate}
\end{definition}
\>

\begin{definition}
If in addition to definition of ring, $a \times b = b \times a \forall a, b \in R$, then $R$ is a \textit{{\color{blue} commutative ring}}.\\
\end{definition}
\>

\begin{definition}
The ring $R$ has a \textit{{\color{blue} multiplicative identity}} if there is an element $1_R \in R$ such that $1_R \times a = a \times 1_R = a$, $\forall a \in R$.\\
The ring $R$ has a \textit{{\color{blue} additive identity}} if there is an element $0_R \in R$ such that $a-b = a+(-b) = 0_R$, where $-b$ is the \textit{{\color{blue} additive inverse}}.\\
\end{definition}
\>

\begin{definition}
A \textit{{\color{blue} division ring}} $R$ is a ring such that:
\begin{enumerate}[label=(\roman*),leftmargin=0pt, itemindent=4em, align=left]
\item $R$ has a multiplicative identity $1_R$;
\item $1_R \neq 0_R$; and
\item $\forall$ nonzero element $a \in R \textbackslash \{0\}$ has a unique multiplicative inverse $a \textsuperscript{-1}$ such that $aa \textsuperscript{-1} = 1 = a \textsuperscript{-1} a$
\end{enumerate}
\end{definition}
\>

\begin{definition}
A \textit{{\color{blue} field}} is a division ring which is commutative.\\
If $R$ is a division ring (field), then $(R, \times)$ is a (commutative) \textit{{\color{blue} multiplicative group}}, $R^\times = R \textbackslash \{0\}$.\\
\end{definition}
\>

\begin{definition}
Let $F = (F, +, \times) $ be a field. A nonempty subset $E \subseteq F$ is a \textit{{\color{blue} subfield}} if:
\begin{enumerate}[label=(\roman*),leftmargin=0pt, itemindent=4em, align=left]
\item $(E, +)$ is an additive subgroup of $(F, +)$;
\item $E$ is closed under multiplication $\times$: $a, b \in E \Rightarrow a \times b \in E$;
\item $1_F \in E$; and
\item $a \in E \textbackslash \{0\} \Rightarrow a \textsuperscript{-1} \in E$
\end{enumerate}
\end{definition}
\>

\begin{remark}
The \textit{{\color{blue} trivial ring}} is $\{0\}$.\\
The \textit{{\color{blue} integer ring}} is $(\mathbb{Z}, +, \times )$ with $ 1 $, but is neither a division ring or field.\\
$n \mathbb{Z} = \{ns | s \in \mathbb{Z}\} $ is a subring of $\mathbb{Z}$.\\
$(\mathbb{Z} / n \mathbb{Z}, +, \times) $ is a commutative ring with $ 1 $ for $ n \geq 2 $.\\
\end{remark}
\>

\begin{remark}
The $2$-dimensional vector space $\mathbb{Q}[\sqrt{D}] = \mathbb{Q} + \mathbb{Q}\sqrt{D} = \{a+b\sqrt{D}|a, b \in \mathbb{Q} \}$ with $\mathbb{Q}$-basis $\{1, \sqrt{D} \}$ is a \textit{{\color{blue} Quadratic Field}}.\\
Define $\mathbb{Q}(\sqrt{D}) = \{ \frac{ a+b\sqrt{D} }{ c+d\sqrt{D} } | a, b, c, d \in \mathbb{Q}, c + d\sqrt{D} \neq 0 \} $. Then $\mathbb{Q}(\sqrt{D}) = \mathbb{Q}[\sqrt{D}]$.\\
More generally, for a field $F$, $\mathbb{Q}(F) = \{\frac{\alpha}{\beta} = \alpha \beta \textsuperscript{-1} | \alpha \beta , \in F, \beta \neq 0\} = F$.\\
\end{remark}
\>

\begin{remark}
Let $H=\mathbb{R} + \mathbb{R} i + \mathbb{R} j + \mathbb{R} k = \{a+bi+cj+dk|a,b,c,d \in \mathbb{R}\}$ be the $4$-dimensional vector space over $\mathbb{R}$ with $\mathbb{R}$-basis $(1, i, j, k)$.\\
The multiplication is extended linearly by distributive law: $i^2 = j^2 = k^2 = -1$, $ij=k=-ji$,$jk=i=-kj$, $ki=j=-ik$. Then $H$ is a \textit{{\color{blue} Real Quaternion Ring}}.\\
$H_\mathbb{Q}=\mathbb{Q}+\mathbb{Q}i+\mathbb{Q}j+\mathbb{Q}k=\{a+bi+cj+dk|a, b, c, d\in \mathbb{Q}\}$ is the \textit{{\color{blue} Rational Hamilton Quaternion Ring}}.\\
\end{remark}
\>

\begin{remark}
Let $\mathbb{R}V[x]=\{f:\mathbb{R}\rightarrow\mathbb{R}\}$ be the set of all real-valued functions. Let $x\mapsto c(x) = c$ be a constant function.\\
For $f, g \in \mathbb{R}V[x]$, the natural addition is $x\mapsto (f+g)(x) = f(x)+g(x)$.\\
The multiplication (not composition) is $x\mapsto (fg)(x)=f(x)g(x)$.\\
The $(\mathbb{R}V[x], +, \times)$ is a commutative \textit{{\color{blue} (real valued-function) ring}} with multiplicative identity $1$ being the constant function $1$.\\
\end{remark}
\>

\begin{definition}
Let $R$ be a ring with $1 \neq 0$. An element $u \in R$ is a \textit{{\color{blue} unit}} if it has a multiplicative identity inverse $u'$ such that $uu'=1=u'u$.\\
The \textit{{\color{blue} set of all units}} of $R$ are $U(R)=\{u\in R | u \ is \ a \ unit\}$.\\
The \textit{{\color{blue} multiplicative group of units of the ring}} $R$ is $(U(R), \times)$.\\
\end{definition}
\>

\begin{remark}
More generally, let $X$ be a set and $R$ be a ring. Let $X \textsubscript{to} R := \{f: X \rightarrow R\}$ be the set of all maps between $X$ and $R$. Then for $f, g \in X \textsubscript{to} R $, there are natural addition $f+g$ and multiplication $fg$ ($x\mapsto f(x)g(x)$) as in previous remark.\\
Then $(X \textsubscript{to} R, +, \times)$ is a ring, called the \textit{{\color{blue} $R$-Valued Function Ring}}.\\
If $R$ has $1$ then so does $X \textsubscript{to} R$. If $R$ is commutative then so does $X \textsubscript{to} R$.\\
Every $c\in R$ defines a constant function (an element in $X \textsubscript{to} R$, $c:X\rightarrow R$; $x \mapsto c(x)=c$.\\
Identify $R$ with the subset of $X \textsubscript{to} R$ of constant function. Then $R$ is a subring of $X \textsubscript{to} R$.\\
\end{remark}
\>

\begin{remark}
Let $n \geq 2$. Then $U(\mathbb{Z}/n\mathbb{Z})$ is a commutative multiplicative group of order $\left| U(\mathbb{Z}/n\mathbb{Z}) \right| = \varphi (n)$.
Hence $\varphi (n)$ is the \textit{{\color{blue} Euler's $\varphi$-function}}, $\varphi (n)=\left|\{1 \leq s \leq n | gcd(s, n) = 1\}\right|$.\\
\end{remark}
\>

\begin{definition}
An \textit{{\color{blue} Integral Domain}} is a commutative ring with $1\neq 0$ such that $\forall a, b, \in R, \ ab = 0 \Rightarrow a=0 \ or \ b=0$,
or equivalently, $\forall a, b \in R, \ a \neq 0, \ b \neq 0 \Rightarrow ab \neq 0$.\\
$\mathbb{Z}$ is an integral domain.\\
Every field is an integral domain.\\
\end{definition}
\>

\begin{definition}
Let $R$ be a ring. A nonzero element $a \in R$ is a 	\textit{{\color{blue} zero divisor}} if there is a nonzero $b\in R$ such that either $ab=0$ or $ba=0$.\\
A commutative ring $R$ with $1$ is an integral domain if and only if $R$ as no zero divisors.\\
\end{definition}
\>

\begin{proposition}
Let $R$ be w ring with $1 \neq 0$. Then $R$ is an integral domain if and only if the cancellation law holds: $\forall a, b, c \in R, \ c \neq 0, \ ca=cb \Rightarrow a=b$.\\	
\end{proposition}
\>

\begin{corollary}
Let $R$ be a finite integral domain, i.e., $R$ is an integral domain with the cardinality $\left| R \right| < \infty$. Then $R$ is a field.\\
\end{corollary}
\>

\begin{proposition}
Let $n \geq 2$. Then the following are equivalent:
\begin{enumerate}[label=(\roman*),leftmargin=0pt, itemindent=4em, align=left]
\item $\mathbb{Z}/n\mathbb{Z}$ is a field
\item $\mathbb{Z}/n\mathbb{Z}$ is an integral domain
\item $n$ is a prime
\end{enumerate}
\end{proposition}
\>

\begin{definition}
Let $R$ be a ring. A nonempty subset $S \subseteq R$ is a \textit{{\color{blue} subring}} of $R$ if:
\begin{enumerate}[label=(\roman*),leftmargin=0pt, itemindent=4em, align=left]
\item $(S, +)$ is an additive subgroup of $(R, +)$ and
\item $S$ is closed under multiplication
\end{enumerate}
$\mathbb{Z} \subset \mathbb{Q} \subset \mathbb{R} \subset \mathbb{C} $
\end{definition}
\>

\begin{proposition}
\textit{{\color{blue} (Subring Criterion)}} Let $R$ be a ring and $S \subseteq R$ a nonempty subset. Then the following are equivalent:
\begin{enumerate}[label=(\roman*),leftmargin=0pt, itemindent=4em, align=left]
\item $S$ is a subring of $R$
\item $S$ is closed under subtracting and multiplication: $a,b\in S \Rightarrow ab\in S$; $a-b = a + (-b) \in S$
\end{enumerate}
\end{proposition}
\>

\begin{remark}
Being a subring is a transitive condition. If $R$ is a subring of $S$ and $S$ is a subring of $T$, then $R$ is a subring of $T$.\\
If both $S_i$ are subring of $R$ and $S_1 \subseteq S_2$, then $S_1$ is a subring of $S_2$.\\
\end{remark}
\>

\begin{remark}
\textit{{\color{blue} (Subring without 1)}} If $R$ is a ring with $1 = 1_R$ then a subring $S \subseteq R$ may not contain $1$, i.e., $m\mathbb{Z} = {ms|s\in \mathbb{Z}, \left| m \right| \geq 2}$ is a subring of $\mathbb{Z}$ which does not contain $1$.
\end{remark}
\>

\begin{remark}
\textit{{\color{blue} (Intersection of subrings)}} Let $R_\alpha$ ($\alpha \in \Sigma$) be a (not necessarily finite or countable) collection of subrings of a ring $R$. Then the intersection $\bigcap\limits_{\alpha \in \Sigma} R_\alpha$ is a subring of $R$.\\
Generally, the union of subrings may not be a subring.\\
\end{remark}
\>

\begin{remark}
\textit{{\color{blue} (Union of ascending subrings)}} Let $R_1 \subseteq R_2 \subseteq \cdots $ be an ascending chain of subrings $R_i$ of a ring $R$.
Then the union $\bigcup\limits_{i=1}^{\infty} R_\alpha$ is a subring of $R$.\\
\end{remark}
\>

\begin{remark}
\textit{{\color{blue} (Addition of subrings)}} Let $R$ be a ring and let $R_i$ be subrings of $R$.\\
Then the addition $R_1 + \cdots + R_n$ is closed under subtraction, but may not be closed under multiplication, hence may not be a subring of $R$.\\
\end{remark}
\>

\begin{remark}
\textit{{\color{blue} (Integral domain is a subring of a field)}}\\
Let $F$ be a field. Let $R \subseteq F$ be a subring such that $1 \in R$. Then $R$ is an integral domain.\\
Every integral domain $R$ is a subring of some field $\mathbb{Q}(R)$ (the fractional field of $R$).\\
\end{remark}
\>

\begin{remark}
\textit{{\color{blue} (Product of Rings)}} let $n \geq 1$ and let $R_i = (R_i, +, \times)$ ($i=1, \ldots, n$) be rings.\\
Then the direct product is a ring, $R = R_1 \times \cdots \times R_n$. (The direct product is $(a_1, \ldots, a_n) \times (a'_1, \ldots, a'_n) = (a_1 a'_1, \ldots, a_n a'_n)$.\\
The unit subgroups has the relation $U(R) = U(R_1)\times \cdots \times U(R_n)$\\
\end{remark}
\>

\section{Examples of Rings}

\begin{definition}
The \textit{{\color{blue} (polynomial ring $R[x] over a ring R$)}} is $(R[x], +, \times)$,\\
where $R[x] = \{\sum_{j=0}^{d} b_j x_j | d \geq 0, \ b_j \in \mathbb{R} \}$.\\
There are natural addition and multiplication operations for polynomials.\\
\end{definition}
\>

\begin{remark}
Let $R$ be a commutative ring with $1$. Let $S:= R[x]$ be the polynomial ring over $R$.
\begin{enumerate}[label=(\roman*),leftmargin=0pt, itemindent=4em, align=left]
\item $R$ is a subring of $S$ which consists of constant polynomial functions.
\item $0_S = 0_R$
\item $S$ contains $1=1_S$, and $1_S = 1_R$.
\end{enumerate}
\end{remark}
\>

\begin{proposition}
\textit{{\color{blue} (Polynomial ring over integral domain)}} Let $R$ be an integral domain. Let $f(x), g(x) \in R[x]$. Then
\begin{enumerate}[label=(\roman*),leftmargin=0pt, itemindent=4em, align=left]
\item $deg(f(x)g(x)) = deg(f(x))+deg(g(x))$
\item $U(R[x]) = U(R)$. Namely, $g(x)$ is a unit of $R[x]$ if and only if $g=a_0 \in R$ (constant polynomial) with $a_0$ a unit in $R$.
\item $R[x]$ is an integral domain
\end{enumerate}
\end{proposition}
\>

\begin{remark}
The {{\color{blue} matrix ring of $n \times n$ square matrices with entries in the ring $R$}} is defined as $(M_n(R), +, \times)$, where\\
$M_n(R) = \left\{A=\begin{pmatrix}
a_{11} & a_{12} & \cdots & a_{1n}\\
a_{21} & a_{22} & \cdots & a_{2n}\\
\vdots & \vdots & \ddots & \vdots\\
a_{m1} & a_{m2} & \cdots & a_{mn}\\
\end{pmatrix} | a_{ij} \in R \right\}$\\
If $A = (a_{ij}), B = (b_{ij}) \in M_n(F)$, then $A+B=(a_{ij} + b_{ij}), AB = (c_{ij})$ where $c_{ij} = \sum_{k=1}^{n} a_{ik} b_{kj}$.\\
$A=(a_{ij}) = Diag[a_{11}, \ldots, a_{nn}]$ is a diagonal matrix if $a_{ij} = 0$ ($i \neq j$).\\
$A=(a_{ij}) = Diag(a_1, \ldots, a_n)$ is a scalar matrix if $a_{ii} = a \in R \ \forall i$, and $a_{ij} = 0$ ($i \neq j$).\\
$A=(a_{ij})$ is an upper triangular matrix if $a_{ij}=0$ ($i < j$). The lower triangular matrix is defined similarly.\\
\end{remark}
\>

\begin{remark}
Let $R$ be a ring and $S = M_n(R)$ the matrix ring with entries in $R$. Then
\begin{enumerate}[label=(\roman*),leftmargin=0pt, itemindent=4em, align=left]
\item $0_S=(a_{ij})$ where $a_{ij} = 0$ (the zero matrix)
\item If $R$ has $1 = 1_R$, then $S$ also has $1 = 1_S$ with $1_S = Diag[1_R, \ldots, 1_R]$
\item The set $S_{c_n}(R) = \{Diag[a, \ldots, a] | a_i \in R\}$ of all scalar matrices in $M_n(R)$ is a subring of $M_n(R)$. There is a natural ring isomorphism $R \cong S_{c_n}(R)$.
\item The set $D_n(R) = \{Diag[]a_1, \ldots, a_n | a_i \in R\}$ of all diagonal matrices in $M_n(R)$ is a subring of $M_n(R)$. There is a natural ring isomorphism $D_n(R) \cong R^n := R \times \cdots \times R $ ($n$ times).
\item The set $UT_n(R):= \{(a_{ij} | (a_{ij} \in R, (a_{ij} = 0 (\forall \ i > j))\}$ of all upper triangular matrices in $M_n(R)$ is a subring of $M_n(R)$. Similarly, the set $LT_n(R)$ of all lower triangular matrices in $M_n(R)$ is a subring of $M_n(R)$.
\item If $R$ is a subring of $R$, then $M_n(T)$ is a subring of $M_n(R)$
\item Even if $R$ is commutative, $M_n(R)$ may not be commutative when $n \geq 2$.
\item If $n \geq 2$, then $M_n(R)$ is not an integral domain (even when $R$ is a field).
\end{enumerate}
\end{remark}
\>

\begin{definition}
Let $R$ be a ring with $1$. Set $GL_n(R) := U(M_n(R))$ the set of all units in $M_n(R)$. Then $GL_n(R)$ is a multiplicative group called the \textit{{\color{blue} general linear group of degree $n$ over $R$}}.
\end{definition}
\>

\begin{definition}
Let $R$ be a commutative ring with $1$. Define determinant $det(A) = \left| A \right|$, Let $SL_n(R) := \{A \in M_n(R) | det(A) = 1\}$ be the set of all matrices in $M_n(R)$ with determinants equal to $1$. Then $SL_n(R)$ is a multiplicative subgroup of $GL_n(R)$ called the \textit{{\color{blue} special linear group of degree $n$ over $R$}}.
\end{definition}
\>

\begin{definition}
\textit{{\color{blue} (Group Rings $R[G]$)}}\\
Let $R$	be a commutative ring with $1 \neq 0$. Let $G = \{g_1, \ldots, g_n\}$ be a finite multiplicative group of order $n$.\\
Then $R[G]$ is a \textit{{\color{blue} group ring}}, where $R[G] = Rg_1 + \cdots + Rg_n = \{a_1 g_1 + \cdots + a_n g_n | a_i \in R\}$.\\
Natural addition is defined as $(\sum_{i=1}^{n} a_i g_i)+ (\sum_{i=1}^{n} b_i g_i) = (\sum_{i=1}^{n} (a_i + b_i) g_i)$.\\
Multiplication is defined as $(\sum_{i=1}^{n} a_i g_i) \times (\sum_{j=1}^{n} b_j g_j) := (\sum_{k=1}^{n} c_k g_k)$ where $c_k=\sum_{g_i g_j = g_k} a_i b_j$ with the sum running $\forall (i, j)$ with $g_i g_j = g_k$.\\
\end{definition}
\>

\begin{remark}
Let $R$ be a commutative ring with $1 \neq 0$, $G$ a multiplicative group, and $R[G]$ the group ring. Then
\begin{enumerate}[label=(\roman*),leftmargin=0pt, itemindent=4em, align=left]
\item $R[G]$ is a commutative ring if and only if $G$ is commutative (=abelian) group
\item $R[G]$ has the multiplicative identity $1=1_R e_G$
\end{enumerate}
\end{remark}
\>

\begin{remark}
Let $R[G]$ be a group ring.
\begin{enumerate}[label=(\roman*),leftmargin=0pt, itemindent=4em, align=left]
\item There is a natural injective ring homomorphism $R \rightarrow R[G]$; $r \mapsto r e_G$. Identify $R$ with the image $R e_G$ of this injective homomorphism.
\item For every $g \in G$, the element $1_R g$ is a unit in $R[G]$
\item There is a natural injective group homomorphism $G \rightarrow U(G[R])$; $g \mapsto 1_R g$. Identify $G$ with the image $1_R G$ of this injective homomorphism.
\item If $S$ is a subring of $R$, then $S[G]$ is a subring of $R[G]$. If $H$ is a subgroup of $G$, then $R[H]$ is a subring of $R[G]$.
\item $T = \{\sum_{i=1}^{n} a_i g_i \in R[G] | \sum_{i=1}^{n} a_i = 0\}$ is a subring of $R[G]$ (an ideal of $R[G]$)
\end{enumerate}
\end{remark}
\>

\begin{remark}
When $R$ is a division ring or field, then $R[G]$ (as an additive group) is a vector space over $R$ of dimension equal to $\left| G \right|$ with basis $\{g_1, \ldots, g_n\} = G$.
Hence $R[G] = Rg_1 + \cdots + Rg_n = Rg_1 \oplus \cdots \oplus Rg_n$, the direct sum of $1$-dimensional vector subspaces $Rg_i$ over $R$.\\
\end{remark}
\>

\section{Ring Homomorphisms}

\begin{definition}
Let $R, S$ be rings. A map $\varphi: R \rightarrow S$ is a \textit{{\color{blue} ring homomorphism}} if it respects the additive and multiplicative structures.\\
$\varphi(a+b) = \varphi(a) + \varphi(b) \ \forall a, b \in R$\\
$\varphi(ab) = \varphi(a)\varphi(b) \ \forall a, b \in R$\\
\end{definition}
\>

\begin{definition}
Let $R, S$ be rings. A map $\varphi: R \rightarrow S$ is a \textit{{\color{blue} ring isomorphism}} if it is a ring homomorphism and bijective.
This is denoted $\varphi: R \xrightarrow{\sim} S$. Rings $R$ and $S$ is \textit{{\color{blue} isomorphic}}, denoted $R \cong S$ or $R \simeq S$.
\end{definition}
\>

\begin{definition}
The \textit{{\color{blue} kernel}} of a ring homomorphism $\varphi$ is defined as $ker \ \varphi = \varphi^{-1}(0_S) = \{a \in R | \varphi(a) = 0_S\} $.
\end{definition}
\>

\begin{remark}
\textit{{\color{blue} (Examples of homomorphism)}}\\
\begin{enumerate}[label=(\roman*),leftmargin=0pt, itemindent=4em, align=left]
\item Let $R, S$ be rings. The map $R \rightarrow S$, $a \mapsto 0$ is a \textit{{\color{blue} zero or trivial map / homomorphism}}.
\item Suppose $R_1$ is a subring of a ring $R$. The map $\iota: R_1 \rightarrow R$, $a \mapsto a$ is a \textit{{\color{blue} inclusion homomorphism}}.\\
\item Let $n \in \mathbb{Z}$. The quotient map $\mathbb{Z} \rightarrow \mathbb{Z}/n\mathbb{Z}$, $s \mapsto \bar{s} = [s]_n $ is a \textit{{\color{blue} quotient homomorphism}} between additive groups $(\mathbb{Z}, +)$ and $(\mathbb(Z)/n\mathbb(Z), +)$.
\item Let $X$ be a set, $R$ a ring, and $X_{to}R = \{f: X \rightarrow R\}$ the ring of all maps from $X$ to $R$. Fix an element $c\in R$. Then $E_c: X_{to}R \rightarrow R$; $f \mapsto E_c(f) := f(c)$ is a \textit{{\color{blue} function evaluation map}}, called the \textit{{\color{blue} Evaluation at $c$}}.
\end{enumerate}
\end{remark}
\>

\begin{proposition}
Let $R, S$ be rings and $\varphi: R \rightarrow S$ be a ring homomorphism. Let $R_1 \subseteq R$ be a subring. Then
\begin{enumerate}[label=(\roman*),leftmargin=0pt, itemindent=4em, align=left]
\item The $\varphi$-image $\varphi(R_1) = {b \in S | b = \varphi(a) \ for \ some \ a \in R_1}$ is a subring of $S$.
\item $ker \ \varphi$ is a subring of $R$ such that $\forall a \in R, \ \forall k \in ker \ \varphi \Rightarrow ak \in ker \ \varphi$. In other words, $ker \ \varphi$ is a subring of $R$, $R(ker \ \varphi) \subseteq ker \ \varphi$ and $(ker \ \varphi)R \subseteq ker \ \varphi$.
\end{enumerate}
\end{proposition}
\>

\begin{definition}
Let $R$ be a ring, $I \subseteq R$ a subset and $r \in R$. The subset $I \subseteq R$ is a \textit{{\color{blue} left-ideal}} of $R$ if:
\begin{enumerate}[label=(\roman*),leftmargin=0pt, itemindent=4em, align=left]
\item $I$ is a subring of $R$; and
\item $I$ is closed under left multiplication by elements from $R$: $rI \subseteq I$ ($\forall r \in R$), i.e., $RI \subseteq I$.
\end{enumerate}
\end{definition}
\>

\begin{definition}
Let $R$ be a ring, $I \subseteq R$ a subset and $r \in R$. The subset $I \subseteq R$ is a \textit{{\color{blue} right-ideal}} of $R$ if:
\begin{enumerate}[label=(\roman*),leftmargin=0pt, itemindent=4em, align=left]
\item $I$ is a subring of $R$; and
\item $I$ is closed under right multiplication by elements from $R$: $Ir \subseteq I$ ($\forall r \in R$), i.e., $IR \subseteq I$.
\end{enumerate}
\end{definition}
\>

\begin{definition}
Let $R$ be a ring, $I \subseteq R$ a subset and $r \in R$. The subset $I \subseteq R$ is a \textit{{\color{blue} (two-sided) ideal}} of $R$ if is both a left-ideal and right-ideal. In other words, $RI \subseteq I$ and $IR \subseteq I$.
\end{definition}
\>

\begin{proposition}
\textit{{\color{blue} (Ideal Criterion)}} Let $R$ be a ring and $I$ a nonempty subset of $R$. The following are equivalent:
\begin{enumerate}[label=(\roman*),leftmargin=0pt, itemindent=4em, align=left]
\item $I$ is a two-sided ideal of $R$;
\item $\forall r \in R$, $\forall a, b \in R \Rightarrow ra, ar, a-b \in I$
\item (If R is commutative) $\forall r \in R$, $\forall a, b \in R \Rightarrow ra, a-b \in I$
\item (If R is commutative with 1) $\forall r \in R$, $\forall a, b \in R \Rightarrow a+rb \in I$
\end{enumerate}
\end{proposition}
\>

\begin{proposition}
Let $R_\alpha$ ($\alpha \in \sum$) be a family of subrings of a ring $R$. Let $J_\alpha$ be a left (resp. 2-sided) ideal of $R_\alpha$.\\
Then the intersection $\bigcap\limits_{\alpha \in \sum}J_\alpha$ is a left (resp. 2-sided) ideal of the subring $\bigcap\limits_{\alpha \in \sum}R_\alpha$.\\
\end{proposition}
\>

\begin{corollary}
Let $J_\alpha$ ($\alpha \in \sum$) be a family of left (resp. 2-sided) ideals of a ring $R$. Then the intersection $\bigcap\limits_{\alpha \in \sum}J_\alpha$ is also a left (resp. 2-sided) ideal of $R$.\\
\end{corollary}
\>

\begin{proposition}
Let $J_\alpha$ ($\alpha \in \sum$) be a finite family of left (resp. 2-sided) ideals of a ring $R$. Then the addition $\sum_{\alpha \in \sum} J_\alpha$ is also a left (resp. 2-sided ideal) of $R$.\\
More generally, if $J_\alpha$ ($\alpha \in \sum$) is an infinite (countable or uncountable) family of left (resp. 2-sided) ideals of a ring $R$, then the subset $\{\sum x_\alpha | x_\alpha \in J_\alpha, x_\alpha \neq 0 \ for \ only \ finitely \ many \ \alpha\}$ is also a left (resp. 2-sided) ideal of $R$.\\
\end{proposition}
\>

\begin{definition}
Let $X$ be a subset of a ring $R$. Let $J_\alpha$ ($\alpha \in \sum$) be all the ideals of $R$ with $J_\alpha \supseteq X$.\\
Then the intersection $\bigcap\limits_{\alpha \in \sum}J_\alpha$ is the \textit{{\color{blue} ideal generated by $X$}}, denoted $(X)$.\\
This $(X)$ is the smallest among all ideals of $R$ containing $X$.\\
If $X=\{r_1, \ldots, r_n\}$, then write $(X)=(r_1, \ldots, r_n)$.\\
\end{definition}
\>

\begin{definition}
For $r \in R$, the ideal $(r)$ generated by a single element $r$ is the \textit{{\color{blue} principal ideal of ring $R$}}.\\
\end{definition}
\>

\begin{definition}
Let $R$ be a ring. An ideal $I$ is \textit{{\color{blue} finitely generated}} if $I=(r_1, \ldots, r_n)$ for some $r_i \in R$.\\
\end{definition}
\>

\begin{proposition}
Let $R$ be a ring; $X, Y, X_i$ the subsets of $R$; and $r_j \in R$.
\begin{enumerate}[label=(\roman*),leftmargin=0pt, itemindent=4em, align=left]
\item Let $J$ be an ideal of $R$. Then $(X) \subseteq J$ if and only if $X \subseteq J$.
\item The equality of ideals holds: $(X_1 \cup \cdots \cup X_n) = (X_1) + \cdots + (X_n)$.
\item In particular, $(r_1, \ldots, r_n) = (r_1) + \cdots + (r_n)$.
\end{enumerate}
\end{proposition}
\>

\begin{proposition}
Let $R$ be a ring. Let $B\subseteq R$ and $a, a_1, \ldots, a_n \in R$.
\begin{enumerate}[label=(\roman*),leftmargin=0pt, itemindent=4em, align=left]
\item $RB = \left\{\sum_{i=1}^{s} r_i b_i | r_i \in R, b_i \in B, s \geq 1 \right\}$ is a left-ideal of $R$, but may not be a 2-sided ideal.
\item More generally, $R\{a_1, \ldots, a_n\} = Ra_1 + \cdots + Ra_n = \left\{\sum_{i=1}^{n} r_i a_i | r_i \in R \right\}$ are left-ideals of $R$, but they may not be 2-sided ideals.
\item The ideal $(a)$ generated by $a$ is given by $(a)=\mathbb{Z}a + aR + Ra + RaR$. An arbitrary element of $(a)$ is of the form $ma + ar + r'a + \sum_{i=1}^{n} r_i a r'_i$ where $m\in \mathbb{Z}; r, r', r_i, r'_i \in R; n \geq 1$.
\item If $R$ contains $1$, then $(a)=RaR$, and an arbitrary element of $(a)$ is of the form $\sum_{i=1}^{n} r_i a r'_i$  where $r_i, r'_i \in R; n \geq 1$.
\item If $R$ is commutative and contains $1$, then $(a)=aR=Ra={ra|r\in R}$. An arbitrary element of $(a)$ is of the form $ra$ where $r \in R$.
\end{enumerate}
\end{proposition}
\>

\begin{proposition}
Let $R$ be a ring with $1 \neq 0$ and $I$ an ideal of $R$. Then the following are equivalent:
\begin{enumerate}[label=(\roman*),leftmargin=0pt, itemindent=4em, align=left]
\item $I = R$
\item $1 \in I$
\item $I$ contains a unit.	
\end{enumerate}
\end{proposition}
\>

\begin{proposition}
Suppose $R$ is a ring with $1$. Let $X \subseteq R$ be a subset, and $b_1, \ldots, b_n \in R$. Then
\begin{enumerate}[label=(\roman*),leftmargin=0pt, itemindent=4em, align=left]
\item the ideal generated by $X$ is $(X)=RXR=\left\{ \sum_{i=1}^{s} r_i a_i r'_i | a_i \in X; r_i, r'_i \in R; s \geq 1\right\}$, the smallest among all ideals of $R$ containing $X$.
\item the ideal generated by $\{b_1, \ldots, b_n\}$ is given by $(b_1, \ldots, b_n) = (b_1) + \cdots + (b_n) = Rb_1R + \cdots + Rb_nR$, the smallest among all ideals of $R$ containing $\{b_1, \ldots, b_n\}$.
\end{enumerate}
\end{proposition}
\>

\begin{proposition}
Let $J_\alpha$ ($\alpha \in \sum$) be a family of left (resp. 2-sided) ideals of a ring $R$.\\
Then the inclusion is $R(\bigcup\limits_{\alpha \in \sum}J_{\alpha}) \subseteq \left\{\sum_{\alpha \in \sum} a_{\alpha} | a_{\alpha} \in J_\alpha; a_{\alpha} \neq 0 \ for \ only \ finitely \ many \alpha \right\}$, where the RHS is a left (resp. 2-sided) ideal of $R$, and the smallest among those of $R$ containing all $J_\alpha$, where LHS = RHS when $R$ contains $1$.\\
If $R$ contains $1$ and $\sum$ is finite, then $R(\bigcup\limits_{\alpha \in \sum}J_{\alpha}) = \sum_{\alpha \in \sum} J_{\alpha}$.\\
\end{proposition}
\>

\begin{proposition}
Let $J, J_1, \ldots, J_n$ be ideals of a ring.\\
Then $J_1 \cdots J_n = \left\{ \sum_{l=1}^{k} a_1(l) \cdots a_n(l) | a_i(l) \in J_i, k \geq 1 \right\}$ and it is an ideal of $R$.\\
in particular, $J^n = J \cdots J = \left\{ \sum_{l=1}^{k} a_1(l) \cdots a_n(l) | a_i(l) \in J, k \geq 1 \right\}$ and it is an ideal of $R$.\\
\end{proposition}
\>

\begin{proposition}
Let $R = R_1 \times \cdots \times R_n$ be a direct product of rings.\\
Then $S_i = \{0_{R_1}\} \times \cdots \times \{0_{R_{i-1}}\} \times R_i \times \{0_{R_{i+1}}\} \times \cdots \times \{0_{R_n}\}$ is an ideal (2-sided) of $R$.\\
Furthermore, $R = \sum_{i=1}^{n} S_i$.\\ 
\end{proposition}
\>

\begin{proposition}
Let $\varphi: R \rightarrow S$ be a ring homomorphism.\\
Then $ker \ \varphi$ is an ideal of $R$.\\
\end{proposition}
\>

\begin{definition}
Let $R$ e a ring and $I \subseteq R$ an ideal. Then $(I, +)$ is a normal subgroup of additive group $(R, +)$. The \textit{{\color{blue} quotient additive group}} is $R/I = \{\bar{r} = r + I | r \in R\}$, with well-defined addition $\bar{r} + \bar{s} := \bar{r+s}$.\\
\end{definition}
\>

\begin{theorem}
Let $R$ be a ring and $I \subseteq R$ an ideal. Then
\begin{enumerate}[label=(\roman*),leftmargin=0pt, itemindent=4em, align=left]
\item for cosets $\bar{r}, \bar{s} \in R/I$, the multiplication $\bar{r} \times \bar{S} := \bar{rs}$ is a well-defined binary operation on $R/I$, i.e., this multiplication does not depend on the choice of representatives $r, s$ of the cosets.
\item $(R/I, +, \times)$ is a ring with $0_{R/I} = \bar{0_R}$.
\item $\bar{r} = 0_{R/I}$ ($=\bar{0_R}$) if and only if $r \in I$.
\end{enumerate}
\end{theorem}
\>

\begin{definition}
Let $R$ be a ring and $I \subseteq R$ an ideal.\\
Then the ring $(R/I, +, \times)$ is the \textit{{\color{blue} quotient ring}} of $R$ by $I$.\\
\end{definition}
\>

\begin{remark}
Let $R$ be a ring and $(I, +)$ a subgroup of the additive group $(R, +)$.\\
Then $I$ is an ideal of $R$ is and only if the multiplication $\times$ on the additive quotient group $(R/I, +)$ is well-defined so that $(R/I, +, \times)$ is a ring.\\	
\end{remark}
\>

\begin{definition}
Let $R$ be a ring, $I \subseteq R$ an ideal, and $R/I$ the quotient ring.\\
The \textit{{\color{blue} surjective quotient map}} $\gamma: R \rightarrow R/I$; $r \mapsto \bar{r} = r + I$ from the additive group $(R, +)$ to the additive group $(R/I, +)$ is a ring homomorphism such that $ker \ \gamma = I$.\\
The \textit{{\color{blue} quotient ring homomorphism}} refers to $\gamma$.\\
\end{definition}
\>

\begin{remark}
\textit{{\color{blue} (Equivalence concepts of kernel and ideal)}}\\
The kernel of every ring homomorphism is an ideal.\\
Every ideal is equal to the kernel of some (surjective) homomorphism.\\
\end{remark}
\>

\begin{definition}
Let $R$ be a commutative ring and $I$ an ideal.\\
An element $a \in R$ is 	\textit{{\color{blue} nilpotent}} if $a^n = 0$ for some $n \geq 1$ (depending on $a$).\\
The set of all nilpotent elements of $R$ is the 	\textit{{\color{blue} nilradical of $R$}}, $nil(R) := \{a \in R | a^n = 0, \ for \ some \ n \geq 1 \}$.\\
In fact, $nil(R)$ is an ideal of $R$, and $nil(R/nil(R)) = 0$.\\
\end{definition}
\>

\begin{definition}
Let $R$ be a commutative ring and $I$ an ideal.\\
The set of \textit{{\color{blue} radical of $I$}} is $rad(I) = \{ r \in R | r^n \in I, \ for \ some \ n \geq 1\}$.\\
In fact, $rad(I)$ is an ideal of $R$ containing $I$ such that $rad(I)/I = nil(R/I)$.\\
\end{definition}
\>

\begin{definition}
Let $R$ be a commutative ring and $J$ an ideal.\\
$J$ is a radical if $rad(J) = J$. Every prime idea of $R$ is ideal.\\
\end{definition}
\>

\begin{definition}
Let $R$ be a commutative ring and $I$ an ideal.
When $R$ contains $1$ and $I \subset R$, \\
define $Jac(I) = \bigcap_{M: max, M \supseteq I} M$, where $M$ runs in the set of all maximal ideals of $R$ containing $I$.\\
In fact, $Jac(I)$ is an ideal of $R$ containing the radical $rad(I)$ of $I$.\\
$Jac(0)$ is the \textit{{\color{blue} Jacobson radical of $R$}}.\\
Thus $Jac(I)$ is the pre-image of $Jac(0_{R/I})$ via $R \rightarrow R/I$.\\
\end{definition}
\>

\begin{remark}
Let $R$ be a commutative ring and $I$ an ideal. Then $nil(R/I^n) \supseteq I/I^n$, and $rad(I^n) \supseteq I$ (the inclusions might be strict).
\end{remark}
\>

\begin{remark}
For the polynomial ring $F[x]$ over field $F$, if $I=(x)$ is the principal ideal generated by $x$, then $I^n = (x^n)$. Hence $nil(F[x]/I^n)=I/I^n$	and $rad(I^n) = I$.
\end{remark}
\>

\begin{remark}
The Jacobson radical of $\mathbb{Z}/12\mathbb{Z}$ is $6\mathbb{Z}/12\mathbb{Z}$, included in the intersection (of two maximal ideals) $(2\mathbb{Z}/12\mathbb{Z}) \cap (3\mathbb{Z}/12\mathbb{Z})$.\\
The Jacobson radical of the polynomial ring $F[x]$ over field $F$ is $0$, which is contained in the intersection (of two maximal ideals) $(x) \cap (x-1)$.\\
\end{remark}
\>

\section{Ring Isomorphisms}

\begin{definition}
\textit{{\color{blue} (First Isomorphism Theorem)}}\\
Let $\varphi: R \rightarrow S$ be a ring homomorphism, $\gamma: R \rightarrow R/ker \ \varphi$ the (surjectve) quotient ring homomorphism, and $\overline{\varphi}: R/ker \ \varphi \ \xrightarrow{\sim} \ \varphi(R)$; $\overline{r} \mapsto \overline{\varphi}(\overline{r}) := \varphi(r)$ a (well-defined) ring homomorphism..\\
Then $\varphi = \overline{\varphi} \circ \gamma$.\\
\begin{equation}\label{diagram}
\begin{tikzcd}
R \arrow{rr}{\varphi} \arrow[swap]{dr}{\gamma} & & \varphi(R) \\[10pt]
    & R/ker \ \varphi \arrow[swap]{ur}{\overline{\varphi}}
\end{tikzcd}
\end{equation}	
\end{definition}
\>

\begin{remark}
Let $R, S$ be a commutative ring with $1$ and $\varphi: R \rightarrow S$ a ring homomorphism. Then $\varphi$ induces a ring homomorphism $\tilde{\varphi}: R[x] \rightarrow S[x]$; $f(x)=\sum a_i x^i \mapsto \widetilde{\varphi}(f(x)) = \sum \varphi(a_i) x_i$.\\
Furthermore, if $J=ker \ \varphi$, then $ker \ \widetilde{\varphi} = J[x] = \left\{\sum_{i=1}^{n} a_i x^i | a_i \in J, n \geq 1 \right\}$	is the polynomial ring with coefficients in $J$.
Finally, $J[x] = J R[x]$ and $J[x]$ is the ideal of $R[x]$ generated by $J$, i.e., $J[x] = (J)$.\\
\end{remark}
\>

\begin{remark}
Let $R$ be a commutative ring with $1$ and $I$ an ideal of $R$.\\
Then there is an isomorphism $R[x]/I[x] \cong (R/I)[x]$.\\	
\end{remark}
\>

\begin{remark}
If $\varphi: R \rightarrow S$ is a ring homomorphism, then it induces a homomorphism (between matrix rings):\\
$\varphi_n: M_n(R) \rightarrow M_n(S)$; $A = (r_{ij}) \mapsto \varphi_n(A) := (\varphi(r_{ij}))$.\\
\end{remark}
\>

\begin{remark}
Let $G={g_1, \l , g_n}$ be a multiplicative group of order $\left| G \right| = n$, $R$ a ring, \\
and $R[G]=Rg_1 + \cdots + Rg_n$ the group ring.\\
Then the map $Tr: R[G] \rightarrow R$; $\sum_{i=1}^{n} r_i g_i \mapsto \sum_{i=1}^{n} r_i$ is a ring isomorphism.\\
\end{remark}
\>

\begin{remark}
\textit{{\color{blue} (One-sided Ideals)}}\\
Let $n \geq 2$ and $M_n(R)$ a matrix ring over a ring $R$. Let $L_k = \{A = (a_{ij}) \in M_n(R) | a_{ij} = 0, \forall j \neq k\}$.\\
Then $L_k$ is a left ideal of $M_n(R)$, but not a right ideal of $M_n(R)$ when $R$ contains $1_R$.\\
Similarly, let $R_k = \{A = (a_{ij}) \in M_n(R) | a_{ij} = 0, \forall i \neq k\}$.\\
Then $R_k$ is a right ideal of $M_n(R)$, but not a left ideal of $M_n(R)$ when $R$ contains $1_R$.\\
\>
More generally, let $1 \leq k_1 < \cdots < k_r \leq n$ with $r < n$.\\
Let $L_{k_1, \ldots, k_r} = \{A = (a_{ij}) \in M_n(R) | a_{ij} = 0, \forall j \notin \{k_1, \ldots, k_r\}\}$.\\
Then $L_{k_1, \ldots, k_r}$ is a left ideal of $M_n(R)$, but not a right ideal of $M_n(R)$ when $R$ contains $1_R$.\\
Let $R_{k_1, \ldots, k_r} = \{A = (a_{ij}) \in M_n(R) | a_{ij} = 0, \forall i \notin \{k_1, \ldots, k_r\}\}$.\\
Then $R_{k_1, \ldots, k_r}$ is a right ideal of $M_n(R)$, but not a left ideal of $M_n(R)$ when $R$ contains $1_R$.\\
\end{remark}
\>

\begin{definition}
\textit{{\color{blue} (Second Isomorphism Theorem)}}\\
Let $R$ be a ring, $R_1 \subseteq R$ a subring, and $J \subseteq R$ an ideal. Then:
\begin{enumerate}[label=(\roman*),leftmargin=0pt, itemindent=4em, align=left]
\item $R_1 + J$ is a subring of $R$
\item $R_1 \cap J$ is an ideal of $R$
\item There is an isomorphism $\varphi: R_1/(R_1 \cap J) \xrightarrow{\sim} (R_1 + J)/J$; $\overline{r} = r + (R_1 \cap J) \mapsto \varphi(\overline(r)) := \overline{r} = r+J$.\\
\end{enumerate}
\end{definition}
\>

\begin{definition}
\textit{{\color{blue} (Third Isomorphism Theorem)}}\\
Let $R$ be a ring, and $I \subseteq J$ ideals of $R$. Then: 
\begin{enumerate}[label=(\roman*),leftmargin=0pt, itemindent=4em, align=left]
\item $J/I$ is an ideal of the quotient ring $R/I$
\item There is an isomorphism $\varphi: R/J \xrightarrow{\sim} (R/I)/(J/I)$; $\overline{r}=r+J \mapsto \overline{r} + J/I = (r+I) + J/I$
\end{enumerate}
\end{definition}
\>

\begin{definition}
\textit{{\color{blue} (Fourth Isomorphism Theorem)}}/Correspondence Theorem for Rings\\
Let $R$ be a ring, $I \subseteq R$ an ideal, and $\gamma: R \rightarrow R/I$ the (surjective) quotient ring homomorphism.\\
Let $\sum_1$ be the set of subrings of $R$ containing $I = ker \ \gamma$, and $\sum_2$ be the set of subrings of $R/I$. Then:
\begin{enumerate}[label=(\roman*),leftmargin=0pt, itemindent=4em, align=left]
\item if $R_1 \in \sum_1$, then $\gamma(R_1) = R_1/I \in \sum_2$. Conversely, if $R'_1 \in \sum_2$, then $R'_1 = R_1/I$ with $R_1:=\gamma^{-1}(R'_1) = \{r\in R | \gamma(r) \in R'_1\} \in \sum_1$.
\item The map $f: \sum_1 \rightarrow \sum_2$; $R_1 \mapsto R_1/I$ is a well-defined bijection.
\item $J_1 \in \sum_1$ is an ideal of $R$ if and only if $J_1/I$ is an ideal of $R/I$. If this is the case, then $R/J_1 \cong (R/I)/(J_1/I)$
\item For $R_i \in \sum_1$, $R_1 \subseteq R_2$ holds if and only if $R_1/I \subseteq R_2/I$ holds.
\end{enumerate}
\end{definition}
\>

\section{Ideals, Rings of Fractions, Local Rings}

\section{Euclidean Domains, PID, UFD}

\end{fullwidth}

\newpage

% Chapter 5: Modules
\chapter{Modules}
\begin{fullwidth}
\section{Basic Axioms}

\end{fullwidth}

\newpage

% Chapter 6: Category Theory
\chapter{Category Theory}
\begin{fullwidth}
\section{Basic Axioms}

\end{fullwidth}

\newpage

\printindex

\end{document}